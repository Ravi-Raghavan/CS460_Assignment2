\documentclass{article}
\usepackage{graphicx} % Required for inserting images
\usepackage[section]{placeins}
\usepackage{float}

\title{Assignment 2}
\date{November 10, 2023}
\author{
  Ravi Raghavan\\
  \texttt{rr1133}
  \and
  Michelle Han\\
  \texttt{mmh255}
}
\begin{document}

%%%%%%%%% PART 1 %%%%%%%%%%%%%%%%%%%
\maketitle
\section{Motion Planning for a 2-link Planar Arm}
% SECTION 1.1
\subsection{Sampling Random Collision-Free Configurations}
% SECTION 1.1 FIGURES
\begin{figure}[h!]
     \begin{minipage}{0.48\textwidth}
    \includegraphics[width=\linewidth]{p1.1.1.png}
    \caption{Provided environment}
  \end{minipage}\hfill
  \begin{minipage}{0.48\textwidth}
    \includegraphics[width=\linewidth]{p1.1.2.png}
    \caption{Assignment 1 environment}
  \end{minipage}\hfill
\end{figure}
% SECTION 1.2
\subsection{Nearest Neighbors}
To find the nearest neighbors, the distance was measured using the end-effector positions. For implementation, forward kinematics was used to determine the coordinates of the end-effectors and the Eucliean distance was calculated from the coordinates.
% SECTION 1.2 FIGURES
\begin{figure}[htbp]
  \centering
  \begin{minipage}{0.45\textwidth}
    \includegraphics[width=\linewidth]{p1.2.1.png}
    \caption{target = [0,0], k=3}
  \end{minipage}\hfill
  \begin{minipage}{0.45\textwidth}
    \includegraphics[width=\linewidth]{p1.2.2.png}
    \caption{target = [3.14,1.5], k=6}
  \end{minipage}
  \begin{minipage}{0.45\textwidth}
    \includegraphics[width=\linewidth]{p1.2.3.png}
    \caption{target = [4,5.2], k=4}
  \end{minipage}\hfill
  \begin{minipage}{0.45\textwidth}
    \includegraphics[width=\linewidth]{p1.2.4.png}
    \caption{target = [1.5,1.1], k=10}
  \end{minipage}
   \begin{minipage}{0.45\textwidth}
    \includegraphics[width=\linewidth]{p1.2.5.png}
    \caption{target = [5.8,-1.5], k=5}
  \end{minipage}
\end{figure}
\subsubsection{With Linear Search Approach}
To implement the Linear Search,
% SECTION 1.2 -> EXTRA CREDIT
\subsubsection{With KD-Tree Search Approach}
To implement the KD-Tree Search,
\subsubsection{Search Comparison}
To compare running times between linear search and kd-tree search, the map used was arm\_polygons.npy and the target was [3.14, 0]. When comparing the running time as the number of configurations increased, the k nearest neighbors was set to 3. When comparing the running time as the number of neighbors increased, the number of random configurations was 100. 
\begin{figure}[]
  \centering
    \includegraphics[width=\linewidth]{comp_k.png}
    \caption{Running time vs. number of nearest neighbors}
    \includegraphics[width=\linewidth]{comp_configs.png}
    \caption{Running time vs. number of configurations}
\end{figure}
\subsubsection{Visualizations}
\begin{figure}[htbp]
  \centering
  \begin{minipage}{0.45\textwidth}
    \includegraphics[width=\linewidth]{p1.2.ec100.png}
    \caption{}
  \end{minipage}\hfill
  \begin{minipage}{0.45\textwidth}
    \includegraphics[width=\linewidth]{p1.2.ec500.png}
    \caption{}
  \end{minipage}
  \begin{minipage}{0.45\textwidth}
    \includegraphics[width=\linewidth]{p1.2.ec1000.png}
    \caption{}
  \end{minipage}\hfill
  \begin{minipage}{0.45\textwidth}
    \includegraphics[width=\linewidth]{p1.2.ec2000.png}
    \caption{}
  \end{minipage}
\end{figure}
\begin{figure}[htbp]
  \centering
  \begin{minipage}{0.45\textwidth}
    \includegraphics[width=\linewidth]{p1.2.ec3n.png}
    \caption{}
  \end{minipage}\hfill
  \begin{minipage}{0.45\textwidth}
    \includegraphics[width=\linewidth]{p1.2.ec5n.png}
    \caption{}
  \end{minipage}
  \begin{minipage}{0.45\textwidth}
    \includegraphics[width=\linewidth]{p1.2.ec10n.png}
    \caption{}
  \end{minipage}\hfill
  \begin{minipage}{0.45\textwidth}
    \includegraphics[width=\linewidth]{p1.2.ec20n.png}
    \caption{}
  \end{minipage}
\end{figure}
% SECTION 1.3
\subsection{Interpolation Along the Straight Line in the C-Space}
describe implementation and resulting visualizations
% SECTION 1.4
\subsection{RRT Implementation}
explain RRT implementation
how did i decide to add an edge to tree
% SECTION 1.5
\subsection{PRM Implementation}
explain PRM implementation
% SECTION 1.6 -> EXTRA CREDIT
\subsubsection{PRM*}
explain PRMstar and how i changed PRM
\subsubsection{Planner Comparison}
%%%%%%%%% PART 2 %%%%%%%%%%%%%%%%%%%
\maketitle
\section{Motion Planning for a Rigid Body in 2D}
% SECTION 2.1
\subsection{Sampling Random Collision-Free Configurations}
\begin{figure}[h!]
	\includegraphics[width= 0.9 \linewidth]{P2_collision_free(1).png}
	\centering
	\caption{Collision-Free Sampling for Environment 1}
	\label{P2_collision_free(1).png}
\end{figure}

\begin{figure}[h!]
	\includegraphics[width= 0.9 \linewidth]{P2_collision_free(2).png}
	\centering
	\caption{Collision-Free Sampling for Environment 2}
	\label{P2_collision_free(2).png}
\end{figure}

\newpage 
\subsubsection{Analysis}
For Problem 2, we know that our configuration $q$ for the 2D Rigid Body can be represented as $(x, y, \theta)$ where $(x, y)$ are the coordinates of the rigid body's geometric center and $\theta$ is the degree of rotation of the rigid body with respect to its center. 

Hence, to obtain 5 samples of random collision-free configurations, I followed the following basic steps 5 times! : 
\begin{itemize}
    \item I sampled $x_{rand}$ uniformly from the range $(0, 2)$, I sampled $y_{rand}$ uniformly from the range $(0, 2)$, and I sampled $\theta_{rand}$ uniformly from the range $(-\pi, \pi)$
    \item The configuration I obtained was $(x_{rand}, y_{rand}, \theta_{rand})$
    \item To determine whether this configuration was in the Free Space of the Configuration Space, I converted the configuration to workspace coordinates and applied collision checking(same algorithms I used in Assignment 1)
    \item If the configuration was in the Free Space of the Configuration Space, I kept it
\end{itemize}

\subsubsection{Collision Detection Review}
To implement collision detection, my methodology can be broken down into two major portions. Let's say we are trying to detect if Polygon P and Q collide. The first thing I did was to construct bounding boxes around each polygon. Essentially, a bounding box for a polygon is akin to drawing a rectangle around the polygon that covers all its corners \newline 

Once the bounding boxes are constructed, when checking for collision, the first thing I check to see is if the bounding boxes for P and Q intersect. If no intersection is detected, I simply return False to indicate that there is NO collision. \newline 

However, if there is a collision detected between the bounding boxes of P and Q, I need to verify this collision. To do this verification, I use the Separating Axis Theorem(SAT). \newline 

Essentially, the main principle behind the Separating Axis Theorem is as follows. If a line/axis exists such that the projections of two polygons onto this line don't overlap, then the polygons don't collide. \newline 

For each edge of both polygons, I computed the normal vector to that edge. Then, I projected each polygon onto that normal vector. If there was a gap in the projections(i.e. there was no overlap in the projections), then I could conclude that there was NO intersection between the two polygons. 

If there was no such gap, I would have to keep continuing this process until I found a normal vector where there was a gap in the projections of the polygons onto this vector. 

At the end of the process, if I did not find any such gap of the projections onto any of the normal vectors, I could conclude that there was a collision between the polygons 

% SECTION 2.2
\newpage 
\subsection{Nearest Neighbors with Linear Search Approach}
\begin{figure}[h!]
	\includegraphics[width= 0.9 \linewidth]{P2_NearestNeighbor(1).png}
	\centering
	\caption{Nearest Neighbors(1)}
	\label{P2_NearestNeighbor(1).png}
\end{figure}

\underline{Program Input}: Target Configuration was $[1, 1, 0]$, number of nearest neighbors was 3, and the input list was specified by rigid$\_$configs.npy
\newpage 
\begin{figure}[h!]
	\includegraphics[width= 0.9 \linewidth]{P2_NearestNeighbor(2).png}
	\centering
	\caption{Nearest Neighbors(2)}
	\label{P2_NearestNeighbor(2).png}
\end{figure}

\underline{Program Input}: Target Configuration was $[0.5, 0.5, 0]$, number of nearest neighbors was 3, and the input list was specified by rigid$\_$configs.npy

\newpage 
\begin{figure}[h!]
	\includegraphics[width= 0.9 \linewidth]{P2_NearestNeighbor(3).png}
	\centering
	\caption{Nearest Neighbors(3)}
	\label{P2_NearestNeighbor(3).png}
\end{figure}

\underline{Program Input}: Target Configuration was $[0.75, 0.5, 0.2]$, number of nearest neighbors was 4, and the input list was specified by rigid$\_$configs.npy

\newpage 
\begin{figure}[h!]
	\includegraphics[width= 0.9 \linewidth]{P2_NearestNeighbor(4).png}
	\centering
	\caption{Nearest Neighbors(4)}
	\label{P2_NearestNeighbor(4).png}
\end{figure}

\underline{Program Input}: Target Configuration was $[1.5, 1.5, -0.2]$, number of nearest neighbors was 4, and the input list was specified by rigid$\_$configs.npy

\newpage
\begin{figure}[h!]
	\includegraphics[width= 0.9 \linewidth]{P2_NearestNeighbor(5).png}
	\centering
	\caption{Nearest Neighbors(5)}
	\label{P2_NearestNeighbor(5).png}
\end{figure}

\underline{Program Input}: Target Configuration was $[1.5, 0.5, -0.5]$, number of nearest neighbors was 5, and the input list was specified by rigid$\_$configs.npy

% SECTION 2.3
\subsection{Interpolation Along the Straight Line in the C-Space}
% SECTION 2.4
\subsection{RRT Implementation}
% SECTION 2.5
\subsection{PRM Implementation}
%%%%%%%%% PART 3 %%%%%%%%%%%%%%%%%%%
\maketitle
\section{Motion Planning for a First-Order Car}
% SECTION 3.1
\subsection{Implementation of Dynamics}
% SECTION 3.2
\subsection{Integration of Dynamics}
% SECTION 3.3
\subsection{RTT and Planning with Dynamics}

\end{document}
